
%% bare_conf.tex
%% V1.3
%% 2007/01/11
%% by Michael Shell
%% See:
%% http://www.michaelshell.org/
%% for current contact information.
%%
%% This is a skeleton file demonstrating the use of IEEEtran.cls
%% (requires IEEEtran.cls version 1.7 or later) with an IEEE conference paper.
%%
%% Support sites:
%% http://www.michaelshell.org/tex/ieeetran/
%% http://www.ctan.org/tex-archive/macros/latex/contrib/IEEEtran/
%% and
%% http://www.ieee.org/

%%*************************************************************************
%% Legal Notice:
%% This code is offered as-is without any warranty either expressed or
%% implied; without even the implied warranty of MERCHANTABILITY or
%% FITNESS FOR A PARTICULAR PURPOSE! 
%% User assumes all risk.
%% In no event shall IEEE or any contributor to this code be liable for
%% any damages or losses, including, but not limited to, incidental,
%% consequential, or any other damages, resulting from the use or misuse
%% of any information contained here.
%%
%% All comments are the opinions of their respective authors and are not
%% necessarily endorsed by the IEEE.
%%
%% This work is distributed under the LaTeX Project Public License (LPPL)
%% ( http://www.latex-project.org/ ) version 1.3, and may be freely used,
%% distributed and modified. A copy of the LPPL, version 1.3, is included
%% in the base LaTeX documentation of all distributions of LaTeX released
%% 2003/12/01 or later.
%% Retain all contribution notices and credits.
%% ** Modified files should be clearly indicated as such, including  **
%% ** renaming them and changing author support contact information. **
%%
%% File list of work: IEEEtran.cls, IEEEtran_HOWTO.pdf, bare_adv.tex,
%%                    bare_conf.tex, bare_jrnl.tex, bare_jrnl_compsoc.tex
%%*************************************************************************

% *** Authors should verify (and, if needed, correct) their LaTeX system  ***
% *** with the testflow diagnostic prior to trusting their LaTeX platform ***
% *** with production work. IEEE's font choices can trigger bugs that do  ***
% *** not appear when using other class files.                            ***
% The testflow support page is at:
% http://www.michaelshell.org/tex/testflow/



% Note that the a4paper option is mainly intended so that authors in
% countries using A4 can easily print to A4 and see how their papers will
% look in print - the typesetting of the document will not typically be
% affected with changes in paper size (but the bottom and side margins will).
% Use the testflow package mentioned above to verify correct handling of
% both paper sizes by the user's LaTeX system.
%
% Also note that the "draftcls" or "draftclsnofoot", not "draft", option
% should be used if it is desired that the figures are to be displayed in
% draft mode.
%
\documentclass[conference]{IEEEtran}
% Add the compsoc option for Computer Society conferences.
%
% If IEEEtran.cls has not been installed into the LaTeX system files,
% manually specify the path to it like:
% \documentclass[conference]{../sty/IEEEtran}





% Some very useful LaTeX packages include:
% (uncomment the ones you want to load)


% *** MISC UTILITY PACKAGES ***
%
%\usepackage{ifpdf}
% Heiko Oberdiek's ifpdf.sty is very useful if you need conditional
% compilation based on whether the output is pdf or dvi.
% usage:
% \ifpdf
%   % pdf code
% \else
%   % dvi code
% \fi
% The latest version of ifpdf.sty can be obtained from:
% http://www.ctan.org/tex-archive/macros/latex/contrib/oberdiek/
% Also, note that IEEEtran.cls V1.7 and later provides a builtin
% \ifCLASSINFOpdf conditional that works the same way.
% When switching from latex to pdflatex and vice-versa, the compiler may
% have to be run twice to clear warning/error messages.






% *** CITATION PACKAGES ***
%
%\usepackage{cite}
% cite.sty was written by Donald Arseneau
% V1.6 and later of IEEEtran pre-defines the format of the cite.sty package
% \cite{} output to follow that of IEEE. Loading the cite package will
% result in citation numbers being automatically sorted and properly
% "compressed/ranged". e.g., [1], [9], [2], [7], [5], [6] without using
% cite.sty will become [1], [2], [5]--[7], [9] using cite.sty. cite.sty's
% \cite will automatically add leading space, if needed. Use cite.sty's
% noadjust option (cite.sty V3.8 and later) if you want to turn this off.
% cite.sty is already installed on most LaTeX systems. Be sure and use
% version 4.0 (2003-05-27) and later if using hyperref.sty. cite.sty does
% not currently provide for hyperlinked citations.
% The latest version can be obtained at:
% http://www.ctan.org/tex-archive/macros/latex/contrib/cite/
% The documentation is contained in the cite.sty file itself.






% *** GRAPHICS RELATED PACKAGES ***
%
\ifCLASSINFOpdf
   \usepackage[pdftex]{graphicx}
  % declare the path(s) where your graphic files are
  % \graphicspath{{../pdf/}{../jpeg/}}
  % and their extensions so you won't have to specify these with
  % every instance of \includegraphics
  % \DeclareGraphicsExtensions{.pdf,.jpeg,.png}
\else
  % or other class option (dvipsone, dvipdf, if not using dvips). graphicx
  % will default to the driver specified in the system graphics.cfg if no
  % driver is specified.
  % \usepackage[dvips]{graphicx}
  % declare the path(s) where your graphic files are
  % \graphicspath{{../eps/}}
  % and their extensions so you won't have to specify these with
  % every instance of \includegraphics
  % \DeclareGraphicsExtensions{.eps}
\fi
% graphicx was written by David Carlisle and Sebastian Rahtz. It is
% required if you want graphics, photos, etc. graphicx.sty is already
% installed on most LaTeX systems. The latest version and documentation can
% be obtained at: 
% http://www.ctan.org/tex-archive/macros/latex/required/graphics/
% Another good source of documentation is "Using Imported Graphics in
% LaTeX2e" by Keith Reckdahl which can be found as epslatex.ps or
% epslatex.pdf at: http://www.ctan.org/tex-archive/info/
%
% latex, and pdflatex in dvi mode, support graphics in encapsulated
% postscript (.eps) format. pdflatex in pdf mode supports graphics
% in .pdf, .jpeg, .png and .mps (metapost) formats. Users should ensure
% that all non-photo figures use a vector format (.eps, .pdf, .mps) and
% not a bitmapped formats (.jpeg, .png). IEEE frowns on bitmapped formats
% which can result in "jaggedy"/blurry rendering of lines and letters as
% well as large increases in file sizes.
%
% You can find documentation about the pdfTeX application at:
% http://www.tug.org/applications/pdftex





% *** MATH PACKAGES ***
%
%\usepackage[cmex10]{amsmath}
% A popular package from the American Mathematical Society that provides
% many useful and powerful commands for dealing with mathematics. If using
% it, be sure to load this package with the cmex10 option to ensure that
% only type 1 fonts will utilized at all point sizes. Without this option,
% it is possible that some math symbols, particularly those within
% footnotes, will be rendered in bitmap form which will result in a
% document that can not be IEEE Xplore compliant!
%
% Also, note that the amsmath package sets \interdisplaylinepenalty to 10000
% thus preventing page breaks from occurring within multiline equations. Use:
%\interdisplaylinepenalty=2500
% after loading amsmath to restore such page breaks as IEEEtran.cls normally
% does. amsmath.sty is already installed on most LaTeX systems. The latest
% version and documentation can be obtained at:
% http://www.ctan.org/tex-archive/macros/latex/required/amslatex/math/





% *** SPECIALIZED LIST PACKAGES ***
%
%\usepackage{algorithmic}
% algorithmic.sty was written by Peter Williams and Rogerio Brito.
% This package provides an algorithmic environment fo describing algorithms.
% You can use the algorithmic environment in-text or within a figure
% environment to provide for a floating algorithm. Do NOT use the algorithm
% floating environment provided by algorithm.sty (by the same authors) or
% algorithm2e.sty (by Christophe Fiorio) as IEEE does not use dedicated
% algorithm float types and packages that provide these will not provide
% correct IEEE style captions. The latest version and documentation of
% algorithmic.sty can be obtained at:
% http://www.ctan.org/tex-archive/macros/latex/contrib/algorithms/
% There is also a support site at:
% http://algorithms.berlios.de/index.html
% Also of interest may be the (relatively newer and more customizable)
% algorithmicx.sty package by Szasz Janos:
% http://www.ctan.org/tex-archive/macros/latex/contrib/algorithmicx/

%Harry 5/12/2015: Addeed algorithm packages
 \usepackage{algorithm} 
 \usepackage{algpseudocode}
 \usepackage{pifont}

% *** ALIGNMENT PACKAGES ***
%
%\usepackage{array}
% Frank Mittelbach's and David Carlisle's array.sty patches and improves
% the standard LaTeX2e array and tabular environments to provide better
% appearance and additional user controls. As the default LaTeX2e table
% generation code is lacking to the point of almost being broken with
% respect to the quality of the end results, all users are strongly
% advised to use an enhanced (at the very least that provided by array.sty)
% set of table tools. array.sty is already installed on most systems. The
% latest version and documentation can be obtained at:
% http://www.ctan.org/tex-archive/macros/latex/required/tools/


%\usepackage{mdwmath}
%\usepackage{mdwtab}
% Also highly recommended is Mark Wooding's extremely powerful MDW tools,
% especially mdwmath.sty and mdwtab.sty which are used to format equations
% and tables, respectively. The MDWtools set is already installed on most
% LaTeX systems. The lastest version and documentation is available at:
% http://www.ctan.org/tex-archive/macros/latex/contrib/mdwtools/


% IEEEtran contains the IEEEeqnarray family of commands that can be used to
% generate multiline equations as well as matrices, tables, etc., of high
% quality.


%\usepackage{eqparbox}
% Also of notable interest is Scott Pakin's eqparbox package for creating
% (automatically sized) equal width boxes - aka "natural width parboxes".
% Available at:
% http://www.ctan.org/tex-archive/macros/latex/contrib/eqparbox/





% *** SUBFIGURE PACKAGES ***
%\usepackage[tight,footnotesize]{subfigure}
% subfigure.sty was written by Steven Douglas Cochran. This package makes it
% easy to put subfigures in your figures. e.g., "Figure 1a and 1b". For IEEE
% work, it is a good idea to load it with the tight package option to reduce
% the amount of white space around the subfigures. subfigure.sty is already
% installed on most LaTeX systems. The latest version and documentation can
% be obtained at:
% http://www.ctan.org/tex-archive/obsolete/macros/latex/contrib/subfigure/
% subfigure.sty has been superceeded by subfig.sty.



%\usepackage[caption=false]{caption}
%\usepackage[font=footnotesize]{subfig}
% subfig.sty, also written by Steven Douglas Cochran, is the modern
% replacement for subfigure.sty. However, subfig.sty requires and
% automatically loads Axel Sommerfeldt's caption.sty which will override
% IEEEtran.cls handling of captions and this will result in nonIEEE style
% figure/table captions. To prevent this problem, be sure and preload
% caption.sty with its "caption=false" package option. This is will preserve
% IEEEtran.cls handing of captions. Version 1.3 (2005/06/28) and later 
% (recommended due to many improvements over 1.2) of subfig.sty supports
% the caption=false option directly:
%\usepackage[caption=false,font=footnotesize]{subfig}
%
% The latest version and documentation can be obtained at:
% http://www.ctan.org/tex-archive/macros/latex/contrib/subfig/
% The latest version and documentation of caption.sty can be obtained at:
% http://www.ctan.org/tex-archive/macros/latex/contrib/caption/




% *** FLOAT PACKAGES ***
%
%\usepackage{fixltx2e}
% fixltx2e, the successor to the earlier fix2col.sty, was written by
% Frank Mittelbach and David Carlisle. This package corrects a few problems
% in the LaTeX2e kernel, the most notable of which is that in current
% LaTeX2e releases, the ordering of single and double column floats is not
% guaranteed to be preserved. Thus, an unpatched LaTeX2e can allow a
% single column figure to be placed prior to an earlier double column
% figure. The latest version and documentation can be found at:
% http://www.ctan.org/tex-archive/macros/latex/base/



%\usepackage{stfloats}
% stfloats.sty was written by Sigitas Tolusis. This package gives LaTeX2e
% the ability to do double column floats at the bottom of the page as well
% as the top. (e.g., "\begin{figure*}[!b]" is not normally possible in
% LaTeX2e). It also provides a command:
%\fnbelowfloat
% to enable the placement of footnotes below bottom floats (the standard
% LaTeX2e kernel puts them above bottom floats). This is an invasive package
% which rewrites many portions of the LaTeX2e float routines. It may not work
% with other packages that modify the LaTeX2e float routines. The latest
% version and documentation can be obtained at:
% http://www.ctan.org/tex-archive/macros/latex/contrib/sttools/
% Documentation is contained in the stfloats.sty comments as well as in the
% presfull.pdf file. Do not use the stfloats baselinefloat ability as IEEE
% does not allow \baselineskip to stretch. Authors submitting work to the
% IEEE should note that IEEE rarely uses double column equations and
% that authors should try to avoid such use. Do not be tempted to use the
% cuted.sty or midfloat.sty packages (also by Sigitas Tolusis) as IEEE does
% not format its papers in such ways.





% *** PDF, URL AND HYPERLINK PACKAGES ***
%
%\usepackage{url}
% url.sty was written by Donald Arseneau. It provides better support for
% handling and breaking URLs. url.sty is already installed on most LaTeX
% systems. The latest version can be obtained at:
% http://www.ctan.org/tex-archive/macros/latex/contrib/misc/
% Read the url.sty source comments for usage information. Basically,
% \url{my_url_here}.





% *** Do not adjust lengths that control margins, column widths, etc. ***
% *** Do not use packages that alter fonts (such as pslatex).         ***
% There should be no need to do such things with IEEEtran.cls V1.6 and later.
% (Unless specifically asked to do so by the journal or conference you plan
% to submit to, of course. )


% correct bad hyphenation here
\hyphenation{op-tical net-works semi-conduc-tor}

%Remove this tag to remove ROMAN numeral numbering in lists
\renewcommand{\labelenumi}{\Roman{enumi}. }
\begin{document}
%
% paper title
% can use linebreaks \\ within to get better formatting as desired
%\title{A Unified Framework for Developing\\ 5G Network Optimization Tools}
%\title{A Unified Framework for 5G Network Optimization}
\title{An Experimental Framework \\for Optimization of 5G Networks}
%\title{A Unified   Framework for Optimization\\ of 5G Networks}
%\title{Adaptive Location Based Service \\ Resource Allocation in 5G Networks}
%\title{Reproducible Data Science PoC Development \\for Network Optimization Algorithms}
%\title{Reproducible PoC Simulation Development\\for Network Optimization Algorithms}
% author names and affiliations
% use a multiple column layout for up to three different
% affiliations



%#############insert Authors




% conference papers do not typically use \thanks and this command
% is locked out in conference mode. If really needed, such as for
% the acknowledgment of grants, issue a \IEEEoverridecommandlockouts
% after \documentclass

% for over three affiliations, or if they all won't fit within the width
% of the page, use this alternative format:
% 
%\author{\IEEEauthorblockN{Michael Shell\IEEEauthorrefmark{1},
%Homer Simpson\IEEEauthorrefmark{2},
%James Kirk\IEEEauthorrefmark{3}, 
%Montgomery Scott\IEEEauthorrefmark{3} and
%Eldon Tyrell\IEEEauthorrefmark{4}}
%\IEEEauthorblockA{\IEEEauthorrefmark{1}School of Electrical and Computer Engineering\\
%Georgia Institute of Technology,
%Atlanta, Georgia 30332--0250\\ Email: see http://www.michaelshell.org/contact.html}
%\IEEEauthorblockA{\IEEEauthorrefmark{2}Twentieth Century Fox, Springfield, USA\\
%Email: homer@thesimpsons.com}
%\IEEEauthorblockA{\IEEEauthorrefmark{3}Starfleet Academy, San Francisco, California 96678-2391\\
%Telephone: (800) 555--1212, Fax: (888) 555--1212}
%\IEEEauthorblockA{\IEEEauthorrefmark{4}Tyrell Inc., 123 Replicant Street, Los Angeles, California 90210--4321}}
% use for special paper notices
%\IEEEspecialpapernotice{(Invited Paper)}
% make the title area
\maketitle
\begin{abstract}
%\boldmath
We propose a unified framework for optimization (UFO) of 5G networks.   
\end{abstract}
Keywords: Root Cause Analysis, self-healing, LTE-A 
% IEEEtran.cls defaults to using nonbold math in the Abstract.
% This preserves the distinction between vectors and scalars. However,
% if the conference you are submitting to favors bold math in the abstract,
% then you can use LaTeX's standard command \boldmath at the very start
% of the abstract to achieve this. Many IEEE journals/conferences frown on
% math in the abstract anyway.

% no keywords




% For peer review papers, you can put extra information on the cover
% page as needed:
% \ifCLASSOPTIONpeerreview
% \begin{center} \bfseries EDICS Category: 3-BBND \end{center}
% \fi
%
% For peerreview papers, this IEEEtran command inserts a page break and
% creates the second title. It will be ignored for other modes.
\IEEEpeerreviewmaketitle

\section{Introduction}
In spite of the various improvements in radio network diagnosis procedure is too slow and no longer viable  
 \begin{figure}[!h]
       	        \centering{\includegraphics[width=3.5in]{figs/arca-arch}}
       	        \caption{Manual Root Cause Analysis}
       	        \label{fig:manua_rca}
       	        \end{figure} 


The lack of automation and scarcity of studies on automated fault detection and diagnosis applicable to emerging technologies such as LTE-A and 5G were the main motivational factors of this research. 

While  previous studies have made several contributions to self-healing, we feel that our work will significantly contribute to the field experience and practical solution development knowledge available to the industry and research community on this topic. 
Specifically, this paper makes the following key contributions:
\begin{enumerate}
\item An adaptive root cause analysis  diagnosis algorithm based on Bayesian network theory.
\item A fault ranking algorithm based on Pareto principle used to prioritize the fault fixing order after diagnosis according to severity and business policies. 
\item A reliable RCA result evaluation method based on Chi-squared hypothesis test. 
\end{enumerate}
The rest of the paper is organized as follows: Section  \ref{sec:background} summarizes some common cellular network faults and relevant diagnostic methods investigated. Section \ref{sec:solution} describes  
 the proposed automatic detection and diagnosis solution  while  representative case studies and experimental results are presented in Section \ref{sec:results}. Section \ref{sec:literature} summarizes relevant literature. Finally, Section \ref{sec:conclusion} concludes the paper and gives some details about future plans.

\section{Network Optimization}
\label{sec:background}
\subsection{Common Cellular Network Faults}
One of the main goals of SON is automation.  As a part of SON, self-healing aims at automating fault detection and diagnosis.  This section summarizes a representative set of typical cellular network faults which are currently detected and diagnosed manually, the section also compares some of the popular diagnostic methods which can be applied when developing automated detection and diagnosis solutions.

\section{A Framework for 5G Network Optimization}
%A Unified Framework for Optimization (UFO) of 5G Networks
\section{Current Development}










\begin{table}[!b]
%% increase table row spacing, adjust to taste
\renewcommand{\arraystretch}{1.3}
% if using array.sty, it might be a good idea to tweak the value of
% \extrarowheight as needed to properly center the text within the cells%\caption{An Example of a Table}
\caption{Simulation network setup }
\label{simulation_setup}
\centering 
\begin{tabular}{@{}l|ll@{}}\hline
Parameter  &Assumption& \\\hline
Carrier frequency&2000 MHz\\
Number of eNodeB cell sites &19\\
Number of cells per eNodeB  &3,  total (57 cells)\\
Inter-site distance  &0.5Km\\
Cell layout&Hexagonal grid with wrap-around\\
Path loss model  &L = 128.1 + 37.6 log10 ( R ),  \\
&R is user distance from BS,in Km\\
Penetration loss&10 dB\\
Lognormal shadowing &Log Normal Fading with 10 dB \\
&standard deviation\\
UE distribution  &Uniform density\\
Maximum BS TX power &46dBm\\
Maximum UE TX power &23dBm\\
Minimum UE TX power &-30dBm\\
Antenna gain after cable loss&15 dBi\\
UE Antenna gain&0 dBi\\
White noise power density&-174 dBm/Hz\\
BS noise figure&5 dB\\
UE noise figure&9 dB\\\hline
\end{tabular}
\end{table}

The following KPIs were selected:
\begin{itemize}


\item \emph{AvgUserThroughput}: Calculated as an average of user throughput (Tu) on the basis of their SINR through the following equation \cite{4:A. Gomez-Andrades}:
   \begin{eqnarray}
              T_{u} = {(1-BLER(SINR))}.{D_u\over{TTI}    }      
              \label{eq:6}
              \end{eqnarray} 
where BLER is the block error probability that depends
on the SINR of user u, Du is its data block payload in
bits, and TTI is the transmission time interval.

\end{itemize}
\subsection{Experimental scenarios and results}
\label{results}





\begin{table*}[!t]
%% increase table row spacing, adjust to taste
\renewcommand{\arraystretch}{1.3}
% if using array.sty, it might be a good idea to tweak the value of
% \extrarowheight as needed to properly center the text within the cells%\caption{An Example of a Table}

\caption{Algorithm Parameters}
\label{Algorithm_Parameters}
\centering
%% Some packages, such as MDW tools, offer better commands for making tables
%% than the plain LaTeX2e tabular which is used here.

\begin{tabular}{@{}lccl@{}}
\hline
Parameter &Operator &Operator &Parameter Description\\
&Policy A&Policy B&\\
\hline
Scope  & 57 cells  & 10 cells& Number of cells in test area \\
Exclusion scope  & [ ]   &  [ ] & Ids of cells excluded from analysis \\
Schedule&	Start now & Closed loop	&Schedule	\\			 Action scope name prefix &ARCA\_MAN	&ARCA\_CLOP	& RCA output scope name prefix\\		Cure trigger strategy &	Manual &Automatic	&Mode of triggering SH SON functions	\\									
Data statistical validity&	2 weeks&2 weeks& Minimum period of gathering network measurements	\\
KPI summarization&	Last hour &Last 15min&Measurement frequency	\\\hline
KPI Thresholds&	 && Expert defined thresholds	\\\hline
RSRP [dBm]&	-117& 	-117&Minimum signal strength required to get LTE service\\
RSRQ [dB]&	-20& 	-20&Minimum signal quality required to maintain LTE service\\
SINR [dB]&	-10& 	-20& Required minimum signal to  interference  and noise ratio \\
HOSR [\%]&	95& 	95&  HO threshold \\
Retainability [\%]&	98& 	98&  Retainability threshold \\
AvgCellThroughput [Kbps]&	150& 	150&  Retainability threshold \\
Distance$_{95}$ [\%]&	95& 	95&  Distance$_{95}$ threshold \\\hline

	
							
Adjust pilot power of neighbor cells&disabled &enabled&Neighbor cell optimization flag\\							 neighbor cell pilot power of neighbor cell: \% &-&-9 dB& $neighborCellMaximumPilotPowerAdjustmentRange$ \\			Pilot power step down for neighbor cell dB	&-&0,6&Neighbor cell adjustment step\\										
Pilot power step up for neighbor cell dB&-&0,6&Neighbor cell adjustment step\\
\hline
\end{tabular}
\end{table*}





\begin{table*}[!t]
%% increase table row spacing, adjust to taste
\renewcommand{\arraystretch}{1.3}
% if using array.sty, it might be a good idea to tweak the value of
% \extrarowheight as needed to properly center the text within the cells%\caption{An Example of a Table}
\caption{Adaptive Root Cause Analysis Diagnosis Results}
\label{case_3_4_results}
\centering
\begin{tabular}{@{}l|lllllllllllllll@{}}\hline
CM  &&Run 1&&&&Run 2&&&&Run 3& \\\hline						
Cell Id&LTE-1&LTE-2&LTE-3&LTE-4&LTE-5&LTE-6&LTE-7&LTE-8&LTE-9&LTE-10&LTE-11&LTE-12&LTE-13\\
Cell power&32.0&32.0&33.6&33.6&33.0&31.5&33.0&33.0&-&32.5&32.0&--&--\\
PM&&&&&&&&&&&\\\hline										wcel.cell-downlink-load&70&70&50&40&50&70&50&10&40&40&70&10&--\\
wcel.uplink-interference-level&2&2&2&2&2&2&2&2&2&2&2&2&--\\
wcel.total-number-of-ishos&0&0&0&0&0&0&0&0&0&0&0&0&--\\
wcel.isho-rscp-ecno-share&0&0&0&0&0&0&0&0&0&0&0&0&--\\
wcel.cell-edge-ecno&5&5&5&5&5&5&5&5&5&5&5&5&--\\
wcel.average-active-set-size&1.7&1.7&1.7&1.7&1.7&1.7&1.7&1.7&1.7&1.7&1.2&1.7&--\\
Expert Additional Input&Code&&&&&&&&&& \\\hline									

Code&C3&--&--&--&--&--&--&--&--&--&--&--&--\\
\hline\hline
Diagnosis Code 
&1&3&--&5&20&20&50&5&20&20&50&5&--\\
Diagnosis Abbreviation
&CH&MN&?&PP&OS&SC&50&5&20&20&50&5&OK\
\\\hline\hline
\end{tabular}
\end{table*}






 
\section{Conclusion}
\label{sec:conclusion}
In this paper, we introduced a common solution for automated fault detection and diagnosis called ARCA. 

The research was motivated by our finding that due to continuous increase in size and complexity, modern cellular networks are more prone to errors than ever before resulting in the current  manual detection and diagnosis procedure performed during  RCA becoming too slow, expensive and no longer practical. 

ARCA uses Naive Bayes probabilistic classifier together with the EM algorithm to offer a simple and adaptive solution for automatic root cause analysis suitable for self-healing emerging cellular networks based on SON such as LTE-A and 5G.

Its main benefits include:
\begin{itemize}
\item Continuous and adaptive learning which increases RCA accuracy and expert knowledge reuse and improvement.
\item ability to work with missing and uncertain data
\item High performance and accuracy with large even with large datasets
\end{itemize}
%\item High success in various industries such as medicine, %linguistics and manufacturing

% conference papers do not normally have an appendix
% use section* for acknowledgement

% trigger a \newpage just before the given reference
% number - used to balance the columns on the last page
% adjust value as needed - may need to be readjusted if
% the document is modified later
%\IEEEtriggeratref{8}
% The "triggered" command can be changed if desired:
%\IEEEtriggercmd{\enlargethispage{-5in}}

% references section



%#################Add acknowledgment here




% can use a bibliography generated by BibTeX as a .bbl file
% BibTeX documentation can be easily obtained at:
% http://www.ctan.org/tex-archive/biblio/bibtex/contrib/doc/
% The IEEEtran BibTeX style support page is at:
% http://www.michaelshell.org/tex/ieeetran/bibtex/
%\bibliographystyle{IEEEtran}
% argument is your BibTeX string definitions and bibliography database(s)
%\bibliography{IEEEabrv,../bib/paper}
%
% <OR> manually copy in the resultant .bbl file
% set second argument of \begin to the number of references
% (used to reserve space for the reference number labels box)
\begin{thebibliography}{11}
\bibitem{1:R. Barco} R.~ Barco, P.~Lazaro, P.~ Mun\'oz, A unified framework for self-healing in wireless
networks. \emph{IEEE Communication Magazine}, vol. 50 (12), 134–142 (2012).
\bibitem{2:R. Szilagyi}P.~ Szil\'agyi and S. Nov\'aczki, “An automatic detection and diagnosis framework for mobile communication systems ,”\emph{ IEEE Transactions on Network and Service Management} , vol. 9 , no. 2 , pp. 184 – 197 ,  2012.
\bibitem{3:B. Benware}B.~ Benware, C.~Schuermyerm, M.~ Sharma, ~ “Determining a Failure Root Cause Distribution From a Population of Layout-Aware Scan Diagnosis Results ,”\emph{ IEEE Design \& Test of Computers} , vol. 29 , no. 1,  pp. 8 – 18 , 2012.
\bibitem{4:A. Gomez-Andrades}A. G\'omez-Andrades, P. Mu\'noz, I. Serrano, R. Barco, ~ “Automatic Root Cause Analysis for LTE Networks
Based on Unsupervised Techniques, ”\emph{IEEE Transactions on Vehicular Technology} , vol. 65 , no. 4 , pp. 2369 - 2386 , April 2016.
\bibitem{5:B. Benware}O.~ Iacoboaiea, B.~ Sayrac, S. B. Jemaa, P.~ Bianchi, ~ “SON Conflict Diagnosis in Heterogenous Networks ,”\emph{ PIMRC} ,  pp. 1459 - 1463 , September 2015.
\bibitem{6:B.H Lee}B. H. Lee, ~ “Using Bayes Belief Networks in Industrial FMEA Modeling and Analysis, ”\emph{IEEE proceedings on annual reliability and maintainability symposium}, pp. 7 - 15 ,  2001.
\bibitem{7:C.B. Do}C.~ B.~ Do, S.~Batzoglou,~ “What is the expectation maximization algorithm?, ”\emph{Nature Biotechnology}, vol 26 pp. 897 - 899 ,  2008.

\bibitem{8:Hamalainen}S. H\"am\"al\"ainen , H. Sanneck , C. Sartori ,~ LTE Self-Organising Networks (SON): Network Management Automation for Operational Efficiency . John Wiley~\& Sons , 2011

\bibitem{9:J. Ramiro}J.~Ramiro, K.~Hamied,~ Self-Organizing Networks (SON): Self-Planning, Self-Optimization and Self-Healing for GSM, UMTS and LTE. John Wiley~\& Sons , October 2011

\bibitem{3GPP:Rel-11}3GPP TS 32.521, \emph{"Technical Specification, Release 11"},
v11.1.0, December 2012 
\bibitem{simulator}I. Maravi\'c, \emph{"LTE Simulator source code"}, accessed
from: https://github.com/i-maravic/LTE-Simulator, November 2016, 

\end{thebibliography}
\end{document}
